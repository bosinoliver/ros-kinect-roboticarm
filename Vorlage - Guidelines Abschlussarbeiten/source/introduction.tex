%%%%%%%%%%%%%%%%%%%%%%%%%%%%%%%%%%%%%%%%%%%%%%%%%%%%%%%%%%%%%%%%%%%%%%%
%% Vorlage f�r Abschlussarbeiten                                     %%
%%-------------------------------------------------------------------%%
%% Datei:        introduction.tex                                    %%
%% Beschreibung: Einleitung f�r die Arbeit                           %%
%%               und ein Ausblick KURZ auf ca einer bis max zwei     %%
%%               Seiten zusammengefasst.                             %%
%% Autor: 			 Stefan Herrmann                                     %%
%% Datum:        28.11.2012                                          %%
%% Version:      1.0.0                                               %%
%%%%%%%%%%%%%%%%%%%%%%%%%%%%%%%%%%%%%%%%%%%%%%%%%%%%%%%%%%%%%%%%%%%%%%%

\chapter{Einleitung}
\index{Einleitung} %% Eintrag im Stichwortverzeichnis

\section{Motivation}
Im ersten Weltkrieg waren die gef�hrlichsten Waffen, die chemischen Waffen. Im kalten Krieg hat die Macht des Atoms, und sein Schrecken aus Nagasaki und Hiroshima, einen dritten Weltkrieg fast schon absurd gemacht. Das, im Normalfall, nur Staaten auf diese Waffen Zugriff haben, wird die meisten Menschen beruhigen. Mit den DeepFake Videos, welche immer wieder in den Medien thematisiert werden, hat nun jedermann Zugriff auf etwas das potentiell als Waffe eingesetzt werden kann. Ob es nun eingesetzt wird um einem Konkurrenzunternehmen zu schaden, zum Zwecke der Propaganda, zur Wahlbeeinflussung oder um einen kalten Konflikt zwischen Konfliktparteien wieder anzufachen, die vorstellbaren Szenarien sind hier vielf�ltig. Zu solchen Bedrohungen, werden immer Gegenma�nahmen gesucht. Bei den Atomwaffen wurde mit Abschreckung gearbeitet, da die Folgen eines Atomwaffeneinsatzes bekannt, gef�rchtet und sofort sichtbar sind. Im Falle der DeepFake Videos ist es mit der Abschreckungstaktik eher schwierig, da die Einsatzm�glichkeiten so Vielf�ltig und die Folgen, selbst f�r den Einsetzenden, nicht einsch�tzbar sind. Als geeignete Gegenma�nahmen bleiben die Sensibilisierung der Menschen f�r dieses Thema und die Entwicklung von effektiven Erkennungsverfahren. Durch die ebenfalls stetige Weiterentwicklung der Verfahren zum erstellen von DeepFake Videos, wird ein gewisses Wettr�sten entstehen. Diese Masterarbeit soll einen Beitrag zu der Forschung, an neuen Verfahren zur Erkennung von DeepFake Vdieos, leisten. 

\section{Aufgabenstellung}
Das Ziel dieser Masterarbeit ist es, mit Hilfe von neuen Ans�tzen und Verfahren, DeepFake Videos mit gr��erer Zuverl�ssigkeit und Effizients zu erkennen. Hierzu werden Verfahren ausgearbeitet, erprobt und evaluiert. 



\section{Gliederung}
Die Arbeit besteht aus folgenden Kapiteln:

\begin{itemize}
\item \textbf{Kaptiel 2 - Grundlagen:}\\
In diesem Kapitel wird auf die Grundlagen, zu verwendeten Verfahren sowie auf die genutzten Arten von neuronalen Netzen, eingegangen. Dies soll die Informationen, welche f�r die gesamte Arbeit und speziell f�r die Ausf�hrungen im dritten und vierten Kapitel wichtig sind, bereitstellen.
\item \textbf{Kapitel 3 - Methodik:}\\
Im dritten Kapitel wird die verwendete Soft- und Hardware beschrieben. Zus�tzlich wird das Vorgehen im Deep Learning Gesamtprozess beschrieben.
\item \textbf{Kapitel 4 - Implementation:}\\
In diesem Kapitel wird die Implementation der Verfahren, welche im Kapitel 3 beschrieben wurden, erkl�rt. Zus�tzlich wird die Architektur und die Implementation der verwendeten neuronalen Netze bechrieben.
\item \textbf{Kapitel 5 - Experimente und Evaluation:}\\
Im f�nften Kapitel werden die durchgef�hrten Trainings und Tests beschrieben. Die Ergebnisse der Experimente werden dargestellt und eingeordnet. 
\item \textbf{Kapitel 6 - Diskussion:}\\
Im letzten Kapitel wird die gesamte Arbeit diskutiert und gesammelte Erkenntnisse aufgef�hrt. Am Ende wird noch ein Ausblick gegeben.
\end{itemize}




