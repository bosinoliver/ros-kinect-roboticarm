%%%%%%%%%%%%%%%%%%%%%%%%%%%%%%%%%%%%%%%%%%%%%%%%%%%%%%%%%%%%%%%%%%%%%%%
%% Vorlage f�r Abschlussarbeiten                                     %%
%%-------------------------------------------------------------------%%
%% Datei:        introduction.tex                                    %%
%% Beschreibung: Einleitung f�r die Arbeit                           %%
%%               und ein Ausblick KURZ auf ca einer bis max zwei     %%
%%               Seiten zusammengefasst.                             %%
%% Autor: 			 Stefan Herrmann                                     %%
%% Datum:        28.11.2012                                          %%
%% Version:      1.0.0                                               %%
%%%%%%%%%%%%%%%%%%%%%%%%%%%%%%%%%%%%%%%%%%%%%%%%%%%%%%%%%%%%%%%%%%%%%%%

\chapter{Einleitung}
\index{Einleitung} %% Eintrag im Stichwortverzeichnis

\section{Motivation}
Nach dem Erfolg des Touchscreen ist sicher, dass die intuitive Bedienung von Technologien, einen gro�en Einfluss auf den Erfolg dieser hat.
Selbst in K�chenmaschinen, wie dem ''Thermomix'' von ''Vorwerk'', werden heutzutage Touchscreens verbaut. Da die Robotik, f�r Industrie und auch f�r andere Nutzergruppen, immer wichtiger wird, stellt sich nun die Frage wie eine Intuitive Steuerung hier aussehen k�nnte. Bei den humanoiden Robotern oder Roboterarmen bietet sich hier ganz klar die Gestensteuerung oder auch die Sprachsteuerung an. Mit einer Gestensteuerung k�nnte ein neuer Nutzer, nach nur einer kurzen Einweisung, einen Roboterarm, mit Sicherheit ohne gr��ere Probleme, steuern. Wenn eine intuitive Steuerung auch die Robotik mehr Menschen n�her bringen k�nnte, dann w�rde dies auch die Forschung und Entwicklung, in dem Bereich der Robotik, beschleunigen. Im Falle vom Robot Operating System, w�rde, durch das gr��ere Interesse an der Robotik, die ROS-Community weiter anwachsen und die Weiterentwicklung von ROS vorantreiben. Innerhalb des Instituts 4 f�r Datentechnik und Schaltungstechnik hat sich ROS bereits, als Basis f�r Projekte in der Robotik, bew�hrt. Unter Anderem wurden autonom fahrende Fahrzeuge, mit ROS als Basis, entwickelt. Die im Institut vorhandenen Fahrzeuge werden entweder durch Programmcode oder durch Fernsteuerungen gesteuert. Um diese M�glichkeiten der Steuerung, um die Gestensteuerung, zu erweitern, wird im Zuge dieser Bachelorarbeit eine Gestensteuerung, mithilfe von Stereokameras, mit ROS entwickelt und implementiert. 

\section{Aufgabenstellung}
Die konkrete Aufgabestellung kann in einem Unterkapitel verdeutlicht werden. W�hrend die Einleitung mehr eine allgemeine Hinleitung zum Thema ist, muss hier dann erl�utert werden, welche Probleme bei bestehenden Systemen bestanden, welche Einschr�nkungen existieren oder welche Weiterentwicklungen gew�nscht sind. 

Ein Beispiel w�re, wenn ein Roboterfahrzeug des Instituts 4 unter Belastung ein Fehlverhalten zeigt. Daraus ergibt sich die Aufgabe zu ermitteln, woher dieses Fehlverhalten kommt und wie es behoben werden kann.

\section{Gliederung}




