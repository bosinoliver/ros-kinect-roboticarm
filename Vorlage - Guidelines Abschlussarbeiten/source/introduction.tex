%%%%%%%%%%%%%%%%%%%%%%%%%%%%%%%%%%%%%%%%%%%%%%%%%%%%%%%%%%%%%%%%%%%%%%%
%% Vorlage f�r Abschlussarbeiten                                     %%
%%-------------------------------------------------------------------%%
%% Datei:        introduction.tex                                    %%
%% Beschreibung: Einleitung f�r die Arbeit                           %%
%%               und ein Ausblick KURZ auf ca einer bis max zwei     %%
%%               Seiten zusammengefasst.                             %%
%% Autor: 			 Stefan Herrmann                                     %%
%% Datum:        28.11.2012                                          %%
%% Version:      1.0.0                                               %%
%%%%%%%%%%%%%%%%%%%%%%%%%%%%%%%%%%%%%%%%%%%%%%%%%%%%%%%%%%%%%%%%%%%%%%%

\chapter{Einleitung}
\index{Einleitung} %% Eintrag im Stichwortverzeichnis
Basierend auf den Erfahrungen vergangener Abschlussarbeiten wurde innerhalb des Instituts 4 diese Vorlage f�r Abschlussarbeiten erstellt. Hierbei wird insbesondere auf h�ufig wiederkehrende Fehler eingegangen, die Studenten beim Verfassen von wissenschaftlichen Arbeiten machen. Dabei waren die meisten Fehler eher handwerklicher Natur und lassen sich einfach vermeiden, wenn man sich bereits vor dem Beginn des Schreibens ein paar Gedanken um Aufbau und verwendeter Software macht. 

Das vorliegende Dokument ist zum einen eine Latex Vorlage, die "`lediglich"' gef�llt werden muss. Dabei braucht man sich kaum noch Gedanken um das Layout zu machen oder �ber die technische Umsetzung von Literaturverzeichnis oder Abbildungsverzeichnis. Die Hoffnung ist, dass die Studenten sich voll und ganz auf den Inhalt der Arbeit konzentrieren k�nnen. Explizit soll hier aber darauf hingewiesen werden, dass die Verwendung von Latex nicht verpflichtend ist. Man kann auch diese Vorlage in Word umsetzen. Dies muss aber in eigener Regie erfolgen. Informationen zum Arbeiten mit Latex sind beispielsweise von Kopka zusammengetragen worden \cite{Kopka2002}.

Diese Vorlage orientiert sich an einem "`IMRAD"'-Schema \cite{Day2006}. Die Abk�rzung steht dabei f�r Introduction - Methods - Results - and - Discussion und beschreibt den grundlegenden Aufbau jeder wissenschaftlichen Arbeit. Was die einzelnen Kapitel bedeuten und welcher Inhalt hier zu erwarten ist, wird in den einzelnen Kapiteln verdeutlicht.

Nach dieser allgemeinen Einleitung zu dieser Vorlage sind noch ein paar Worte zu der von ihnen zu verfassenden Einleitung zu verlieren. Generell leitet die Einleitung zu dem Thema oder der Fragestellung hin. Was war der Zweck dieser Arbeit? Welche Fragestellung lag zugrunde oder was sollte mit der Entwicklung erzielt werden. Im Institut~4 haben sich hier Informationen �ber die Herkunft des Roboters\index{Roboter} bew�hrt. Welche Probleme bestanden, Historie der Entwicklung und auch die konkrete Aufgabenstellung. 

\section{Aufgabenstellung}
Die konkrete Aufgabestellung kann in einem Unterkapitel verdeutlicht werden. W�hrend die Einleitung mehr eine allgemeine Hinleitung zum Thema ist, muss hier dann erl�utert werden, welche Probleme bei bestehenden Systemen bestanden, welche Einschr�nkungen existieren oder welche Weiterentwicklungen gew�nscht sind. 

Ein Beispiel w�re, wenn ein Roboterfahrzeug des Instituts~4 unter Belastung ein Fehlverhalten zeigt. Daraus ergibt sich die Aufgabe zu ermitteln, woher dieses Fehlverhalten kommt und wie es behoben werden kann.




