%%%%%%%%%%%%%%%%%%%%%%%%%%%%%%%%%%%%%%%%%%%%%%%%%%%%%%%%%%%%%%%%%%%%%%%
%% Vorlage f�r Abschlussarbeiten                                     %%
%%-------------------------------------------------------------------%%
%% Datei:        introduction.tex                                    %%
%% Beschreibung: Einleitung f�r die Arbeit                           %%
%%               und ein Ausblick KURZ auf ca einer bis max zwei     %%
%%               Seiten zusammengefasst.                             %%
%% Autor: 			 Stefan Herrmann                                     %%
%% Datum:        28.11.2012                                          %%
%% Version:      1.0.0                                               %%
%%%%%%%%%%%%%%%%%%%%%%%%%%%%%%%%%%%%%%%%%%%%%%%%%%%%%%%%%%%%%%%%%%%%%%%

\chapter{Einleitung}
\index{Einleitung} %% Eintrag im Stichwortverzeichnis

\section{Motivation}

\section{Aufgabenstellung}
Die konkrete Aufgabestellung kann in einem Unterkapitel verdeutlicht werden. W�hrend die Einleitung mehr eine allgemeine Hinleitung zum Thema ist, muss hier dann erl�utert werden, welche Probleme bei bestehenden Systemen bestanden, welche Einschr�nkungen existieren oder welche Weiterentwicklungen gew�nscht sind. 

Ein Beispiel w�re, wenn ein Roboterfahrzeug des Instituts~4 unter Belastung ein Fehlverhalten zeigt. Daraus ergibt sich die Aufgabe zu ermitteln, woher dieses Fehlverhalten kommt und wie es behoben werden kann.

\section{Gliederung}




