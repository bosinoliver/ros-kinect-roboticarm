%%%%%%%%%%%%%%%%%%%%%%%%%%%%%%%%%%%%%%%%%%%%%%%%%%%%%%%%%%%%%%%%%%%%%%%
%% Vorlage f�r Abschlussarbeiten                                     %%
%%-------------------------------------------------------------------%%
%% Datei:        introduction.tex                                    %%
%% Beschreibung: Einleitung f�r die Arbeit                           %%
%%               und ein Ausblick KURZ auf ca einer bis max zwei     %%
%%               Seiten zusammengefasst.                             %%
%% Autor: 			 Stefan Herrmann                                     %%
%% Datum:        28.11.2012                                          %%
%% Version:      1.0.0                                               %%
%%%%%%%%%%%%%%%%%%%%%%%%%%%%%%%%%%%%%%%%%%%%%%%%%%%%%%%%%%%%%%%%%%%%%%%

\chapter{Einleitung}
\index{Einleitung} %% Eintrag im Stichwortverzeichnis

\section{Motivation}
Nach dem Erfolg des Touchscreen ist sicher, dass die intuitive Bedienung von Technologien, einen gro�en Einfluss auf den Erfolg dieser hat.
Selbst in K�chenmaschinen, wie dem ''Thermomix'' von ''Vorwerk'', werden heutzutage Touchscreens verbaut. Da Roboter, f�r Industrie und auch f�r andere Nutzergruppen, immer wichtiger werden, stellt sich nun die Frage, wie eine Intuitive Steuerung hier aussehen k�nnte. Bei den humanoiden Robotern oder Roboterarmen bietet sich hier ganz klar die Gestensteuerung oder auch die Sprachsteuerung an. Mit einer Gestensteuerung k�nnte ein neuer Nutzer, nach nur einer kurzen Einweisung, einen Roboterarm, mit Sicherheit ohne gr��ere Probleme, steuern. Wenn eine intuitive Steuerung, auch die Robotik mehr Menschen n�her bringen k�nnte, dann w�rde dies auch die Forschung und Entwicklung, in dem Bereich der Robotik, beschleunigen. Im Falle vom Robot Operating System, w�rde, durch das gr��ere Interesse an der Robotik, die ROS-Community weiter anwachsen und die Weiterentwicklung von ROS vorantreiben. Innerhalb des Instituts 4 f�r Datentechnik und Schaltungstechnik hat sich ROS bereits, als Basis f�r Projekte in der Robotik, bew�hrt. Unter Anderem wurden autonom fahrende Fahrzeuge, mit ROS als Basis, entwickelt. Die im Institut vorhandenen Fahrzeuge werden entweder durch Programmcode oder durch Fernsteuerungen gesteuert. Um diese M�glichkeiten der Steuerung, um die Gestensteuerung, zu erweitern, wird im Zuge dieser Bachelorarbeit eine Gestenerkennung, mit Hilfe von Stereokameras, mit ROS entwickelt und implementiert. 

\section{Aufgabenstellung}
Das Ziel dieser Bachelorarbeit ist es, mit Hilfe einer Stereokamera, mit ROS eine Gestenerkennung zu entwickeln und zu implementieren. Als sekund�r Ziel wird die Entwicklung eines ROS-Paketes, welches ein Beispiel f�r die Anwendung des Gestenerkennungssystems aufzeigt, angestrebt. Der Hauptzweck dieser Arbeit, neben der Entwicklung der Gestenerkennung, ist es, eine erste Grundlage, f�r Entwicklungen und nachfolgende Arbeiten, im Bereich der nat�rlichen Interaktion mit Robotersystemen zu legen. Wie sich die Gestenerkennung in die Umgebung von ROS einf�gt und wie sie in dieser Umgebung verwendet werden kann, sind Kernfragen dieser Grundlage. 


\section{Gliederung}
Die Arbeit besteht aus folgenden Kapiteln:

\begin{itemize}
\item \textbf{Kapitel 2 - Systemaufbau:}\\
Hier werden die Systemkomponenten aufgef�hrt und es werden jeweils kurze Erkl�rungen, zu jeder Komponente, gegeben. Ebenfalls wird hier Software und Hardware die verwendet wurde, aber nicht direkt zum Gestenerkennungssystem geh�rt, aufgef�hrt und kurz erkl�rt. Am Ende des Kapitels wird der gesamte Systemaufbau dargestellt und erkl�rt.
\item \textbf{Kaptiel 3 - Grundlagen:}\\
In diesem Kapitel werden grundlegende Funktionsweisen und Konzepte, der verwendeten Hardware und Software, erl�utert. Es wird hier auf die Stereokamera, das Framework ''OpenNI'', ROS und das Framework ''MoveIt!'' eingegangen.
\item \textbf{Kapitel 4 - Implementation des Gestenerkennungssystems mit ROS:}\\
Hier wird auf die Implementation des Gestenerkennungssystems eingegangen. Es wird ebenfalls auf die Inbetriebnahme und den Test des Systems eingegangen.
\item \textbf{Kapitel 5 - Implementation von Gestensteuerungen basierend auf dem Gestenerkennungssystem:}\\
In diesem Kapitel werden die entwickelten ROS-Pakete vorgestellt, welche Gestensteuerungen, die das Gestenerkennungssystem verwenden, implementiert haben. Es wird auf m�gliche Voraussetzungen, die Inbetriebnahme der Pakete und die Verwendung der Pakete eingegangen.
\item \textbf{Kapitel 6 - Diskussion:}\\
Im letzten Kapitel werden die Ergebnisse erl�utert und bewertet. Es wird zus�tzlich auf aufgetretene Probleme eingegangen. Zudem werden Empfehlungen, f�r weitere Arbeiten und Verbesserungen am System, gegeben. Am Ende wird noch ein allgemeiner Ausblick gegeben.
\end{itemize}




