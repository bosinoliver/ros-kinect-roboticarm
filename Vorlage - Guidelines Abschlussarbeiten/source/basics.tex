%%%%%%%%%%%%%%%%%%%%%%%%%%%%%%%%%%%%%%%%%%%%%%%%%%%%%%%%%%%%%%%%%%%%%%%
%% Vorlage f�r Abschlussarbeiten                                     %%
%%-------------------------------------------------------------------%%
%% Datei:        basics.tex                                         %%
%% Beschreibung: Grundlagenteil welcher verwendete Hard-   %%
%%               und Software n�her beschreibt.                            %%
%% Autor: 			 Stefan Herrmann                                     %%
%% Datum:        04.12.2012                                          %%
%% Version:      1.0.1                                               %%
%%%%%%%%%%%%%%%%%%%%%%%%%%%%%%%%%%%%%%%%%%%%%%%%%%%%%%%%%%%%%%%%%%%%%%%

\chapter{Grundlagen}
\index{Grundlagen} %% Eintrag im Stichwortverzeichnis
In diesem Kapitel wird auf die Grundlagen zu verwendeter Hard- und Software eingegangen. Dies soll Hintergrundinformationen zur gesamten Arbeit und speziell f�r die Ausf�hrungen im vierten Kapitel bereitstellen. Hierbei wird sich nur auf die wesentlichen Komponenten, f�r das Gesamtsystem, beschr�nkt.

\section{Stereokamera}
\index{Stereokamera}
Eine Stereokamera ist ein Kamerasystem,  dass zur Gewinnung von Tiefenbildern genutzt wird. Diese Tiefenbilder enthalten, im Gegensatz zu normalen Farbbildern, den Abstand einzelner Punkte zum Sensor der Kamera. Diese Kamerasysteme haben immer zwei optische Sensoren zur Bilderzeugung. Es gibt Systeme mit zwei RGB-Sensoren und Systeme mit einem RGB-Sensor sowie einem Infrarotsensor.

\subsection{Typen}
Unter den Stereokameras gibt es verschiedene Typen, die sich in zwei Gruppen einteilen lassen: Diese Gruppen sind, Kameras mit passiven Verfahren und Kameras mit aktiven Verfahren. F�r diese Arbeit wurden die Beschreibungen auf die g�ngigsten typen, von Stereokameras, begrenzt.

\subsection*{Embedded Stereo}
\index{Embedded Stereo}
Die ''Embedded Stereo'' Kameras nutzen das passive Stereoverfahren, um Tiefenbilder zu erzeugen. Diese Kamerasysteme haben meist zwei RGB-Sensoren um Bilder aufzunehmen, sowie einen eingebetteten Mikroprozessor oder FPGA, f�r die Berechnung von Tiefenbildern, integriert. Das passive Stereoverfahren ist an das menschliche Sehen angelehnt, wo aus zwei Bildern ein ganzes 3D-Bild gemacht wird. Das passive Verfahren zur Erzeugung der Tiefenbilder l�sst sich in mehrere Schritte aufteilen. Zu erst wird mit beiden RGB-Sensoren gleichzeitig ein Bild aufgenommen. Anschlie�end wird an beiden Bildern eine Merkmalsextraktion durchgef�hrt. Diese Bildmerkmale, auch als ''Keypoints'' bezeichnet, lassen Unterschiede zu Ihrer Umgebung erkennen. Die gefundenen Bildmerkmale, beider Bilder, werden bei der Korrespondezsuche miteinander verglichen. Nachdem die Korrespondenzsuche abgeschlossen ist, ist es noch notwendig falsche Korrespondenzen aus den Ergebnissen herauszufiltern. Mit den endg�ltigen korrespondierenden Bildmerkmalen kann nun, mittels Triangulation, die Entfernung zu diesem Merkmal im Bild errechnet werden.\cite{BLANK2013}\\ Um den Prozessor des nutzenden Hauptsystems zu entlasten, werden die Berechnungen, f�r das beschriebene Verfahren, auf dem eingebetteten Mikroprozessor bzw. FPGA durchgef�hrt. Da der hohe Rechenaufwand auf eine erh�hte Leistungsaufnahme schlie�en l�sst, werden die''Embedded Stereo'' Kameras nicht f�r mobile Systeme empfohlen. 

\subsection*{Time-Of-Flight}
\index{Time-Of-Flight}
Eine''Time-Of-Flight'' Kamera geh�rt zu den Kamerasystemen die aktive Verfahren, zur Generierung von Tiefenbildern, nutzen. Das von ''Time-Of-Flight'' Kameras genutzte Verfahren ist das Laufzeitverfahren. Bei dem Laufzeitverfahren wird ein Signal von der Kamera ausgesendet und die Zeit ermittelt wie lange das Reflektierte Signal gebraucht hat, um wieder an der Kamera aufzutreffen. Aus der Laufzeit t und Geschwindigkeit v des Signals kann, �ber die Formel $S =  v t$, direkt auf die Strecke S zum reflektierenden Objekt geschlossen werden. Als typische Emitter-Sensor Systeme, in dem Laufzeitverfahren, z�hlen Radar-, Ultraschall- und Infrarotsysteme.\cite{SCHMIEDECKE2009}\\ Die ''Time-Of-Flight'' Kameras nutzen in der Regel Infrarotsysteme. Die g�ngigen ''Time-Of-Flight'' Kameras haben meist einen RGB-Sensor, einen oder mehrere Infrarotemitter und einen Infrarotsensor verbaut. Aufgrund der hohen Ausbreitungsgeschwindigkeit von Licht, werden hohe Anforderungen an die Elektronik, zur Zeitmessung, gestellt. Unter anderem legt die kleinste messbare Zeitdifferenz $\triangle$t die maximale Tiefenaufl�sung $\triangle$R, mit der Formel $\triangle R = \frac{v \triangle t}{2}$ , fest.\cite{JIANGBUNKE1997} Durch diese hohen Anforderungen an die Elektronik, werden hochaufl�sende ''Time-Of-Flight'' Kameras schnell sehr kostenintensiv. Da hier Licht, im Infrarotbereich, detektiert wird, sind ''Time-Of-Flight'' Kameras nicht f�r Bereiche mit direkter Sonneneinstrahlung geeignet. In Bereichen mit direkter Sonneneinstrahlung empfehlen sich stattdessen Kameras die auf passive Verfahren zur Tiefengewinnung zur�ckgreifen.

\subsection*{Infrarotmuster}
\index{Infrarotmuster}


\section{OpenNI}
\index{OpenNI}

\section{ROS}
\index{ROS}

\section{MoveIt!}
\index{MoveIt!}