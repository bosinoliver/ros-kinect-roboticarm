%%%%%%%%%%%%%%%%%%%%%%%%%%%%%%%%%%%%%%%%%%%%%%%%%%%%%%%%%%%%%%%%%%%%%%%
%% Vorlage f�r Abschlussarbeiten                                     %%
%%-------------------------------------------------------------------%%
%% Datei:        abstract.tex                                        %%
%% Beschreibung: Kurzzusammenfassung der Arbeit. Hierbei werden      %%
%%               Einleitung, Grundlagen, Messungen, Tests, Ergebnisse%%
%%               und ein Ausblick KURZ auf ca einer bis max zwei     %%
%%               Seiten zusammengefasst.                             %%
%% Autor: 			 Stefan Herrmann                                     %%
%% Datum:        05.12.2012                                          %%
%% Version:      1.1.1                                               %%
%%%%%%%%%%%%%%%%%%%%%%%%%%%%%%%%%%%%%%%%%%%%%%%%%%%%%%%%%%%%%%%%%%%%%%%

\chapter*{Abstrakt} %% Dieses Kapitel taucht nicht im Inhaltsverzeichnis auf.
\index{Abstrakt} %% Eintrag im Stichwortverzeichnis

Dieses Dokument dient zum Einen dazu, eine einheitliche Vorlage f�r alle Abschlussarbeiten im Institut 4 bereitzustellen. Zum Anderen wird es den Studenten hiermit vereinfacht eine umfassende Abschlussarbeit \index{Abschlussarbeit} anzufertigen, die gewissen Grundregeln f�r wissenschaftliche Arbeiten entspricht. Diese Vorlage ist nicht bindend. Es wird allerdings empfohlen, sich hieran zu halten. Auf diese Weise kann man sich auf den Inhalt und das Schreiben konzentrieren, ohne sich gro�e Gedanken �ber das Layout und die technische Umsetzung Gedanken machen zu m�ssen.

Das erste Kapitel einer Arbeit ist das Abstrakt. Auch wenn es chronologisch an erster Stelle steht, so wird es doch eigentlich zeitlich ganz am Ende geschrieben. Denn hier wird dem Leser ein umfassender �berblick \index{�berblick} �ber die Arbeit erm�glicht, ohne diese komplett lesen zu m�ssen. Alle wichtigen Punkte der Ausarbeitung m�ssen dabei aufgef�hrt werden. Beginnend mit einer Einleitung, den Grundlagen (bzw. Methoden), die Ergebnisse und Ausblick bzw. Diskussion. Die Kunst hierbei ist, dass sich \textbf{kurz} zu fassen. Ein Abstrakt sollte maximal zwei Seiten umfassen. 


