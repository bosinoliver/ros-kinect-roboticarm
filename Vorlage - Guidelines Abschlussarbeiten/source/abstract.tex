%%%%%%%%%%%%%%%%%%%%%%%%%%%%%%%%%%%%%%%%%%%%%%%%%%%%%%%%%%%%%%%%%%%%%%%
%% Vorlage f�r Abschlussarbeiten                                     %%
%%-------------------------------------------------------------------%%
%% Datei:        abstract.tex                                        %%
%% Beschreibung: Kurzzusammenfassung der Arbeit. Hierbei werden      %%
%%               Einleitung, Grundlagen, Messungen, Tests, Ergebnisse%%
%%               und ein Ausblick KURZ auf ca einer bis max zwei     %%
%%               Seiten zusammengefasst.                             %%
%% Autor: 			 Stefan Herrmann                                     %%
%% Datum:        05.12.2012                                          %%
%% Version:      1.1.1                                               %%
%%%%%%%%%%%%%%%%%%%%%%%%%%%%%%%%%%%%%%%%%%%%%%%%%%%%%%%%%%%%%%%%%%%%%%%

\chapter*{Abstrakt} %% Dieses Kapitel taucht nicht im Inhaltsverzeichnis auf.
\index{Abstrakt} %% Eintrag im Stichwortverzeichnis
Gegenstand dieser Bachelorarbeit ist die Entwicklung eines Gestenerkennungssystems, mit Hilfe von der Stereokamera ''Kinect'' von ''MICROSOFT'', mit ROS. Es wird zuerst auf alle Komponenten, die den Systemaufbau ausmachen, eingegangen. Das Konzept hinter dem System ist ebenfalls Teil der Ausf�hrungen. Zu den wichtigsten Komponenten, dies sind die Stereokamera, ROS, ''MoveIt!'' und ''OpenNI'', werden danach die Grundlagen erkl�rt. Auf die Beschreibung des Systems und seiner Komponenten folgen die Ausf�hrungen zur Implementation und Inbetriebnahme des Systems. Nachfolgend wird die Implementation und Funktionsweise von zwei Paketen, die im Zuge dieser Bachelorarbeit entwickelt wurden, erkl�rt. Diese Pakete zeigen m�gliche Anwendungen, f�r das Gestenerkennungssystem auf. Die Ergebnisse, Probleme und Erkenntnisse dieser Bachelorarbeit, werden am Schluss diskutiert. Ein Ausblick wird ebenfalls gegeben.\\
F�r Studierende und Interessierte, welche den Einstieg in ROS und die nat�rliche Interaktion suchen, bietet diese Arbeit eine erste Orientierung. 


